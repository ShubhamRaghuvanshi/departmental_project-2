\documentclass[12pt,a4paper]{article}		% Necessary line
\usepackage{graphicx}	
\usepackage{grffile}	% Necessary line
\usepackage{mathtools}
\usepackage{breqn}
\usepackage{float}
\usepackage{caption}
\usepackage{subcaption}
\usepackage{hyperref}


\usepackage[margin=.6in,footskip=0.25in]{geometry}


\title{\textbf { Study of production of leptophobic $Z^\prime$ boson in assosiation with top pair in pp collisions at 14 TeV.} }
 
 
\author{Shubham P. Raghuvanshi\\Department of High Energy Physics, TIFR, Mumbai.}

% Start the document
\begin{document}		% Necessary line: starts the document
\maketitle
\thispagestyle{empty}
\newpage
\tableofcontents
\newpage
\thispagestyle{empty}
\listoffigures
\newpage
\section{Introduction}


The hypothetical $Z^\prime$ boson is required by many extensions of the Standard Model \cite{paul,zpdg}, simplest of these models are U(1) extension of SM which require an additional massive vector boson which couples to standard model fermions. It is electrically-neutral, color singlet spin 1 particle. The mass of $Z^\prime$ in principle depends upon it's coupling to various SM and BSM particles, which are model dependant. Therefore apart from discovering $Z^\prime$ is also to be identified with the correct model.

In this work we analyse events in which $Z^\prime$ is produced in pp collision at $\sqrt{s} = 14$ TeV in assosiation with top anti-top pair \ref{fynnmann}. We chose B-L model \cite{bl} for this study since it is one of the simplest U(1) extensions of the SM. However the methods used for analysis are fairly general. Using MSTW2008lo68cl\_nf3 \cite{pdfsets} parton distribution functions(PDF), with $M_{Z^\prime} = 1$ TeV and $M_t = 172$ GeV, $ \text{MADGRAPH}_5$aMC@NLO \cite{mad1,mad2} provides the cross section for this process to be $ \sigma_{B-L}( p p \to Z^\prime t \bar{t}) = 0.000183 \pm 1.3905 \times 10^{-6}$(stat) pb. In this analysis we will be interested mainly in the $Z^\prime \to t \bar{t}$ decay channel, the final state therefore contains 4 top quarks with the BL cross section of $ \sigma_{B-L}(pp \to Z^\prime t \bar{t} \to t \bar{t}t \bar{t}) = 9.877 \times 10^{-8} \pm  1.5 \times 10^{-10}$ pb. The most dominant background for this is 4 top QCD background with the SM cross section of $\sigma_{SM}(pp \to t \bar{t}t \bar{t}) = 0.01182 \pm 3.19 \times 10^{-5}$ pb. 

	\begin{figure}[h] 	 		
		\begin{centering}	
			\includegraphics[scale=0.2]{./gg2ztt0.png}				\includegraphics[scale=0.2]{./gg2ztt1.png}
			\includegraphics[scale=0.2]{./gg2ztt0.png} \\ 			
			\includegraphics[scale=0.2]{./ss2ztt0.png} 
			\includegraphics[scale=0.2]{./ss2ztt1.png}
			\caption{ LO Fynnman diagrams for the parton level processes $gg \to Z^\prime t \bar{t}$, $qq \to Z^\prime t \bar{t}$. }
			\label{fynnmann}
			\centering
		\end{centering}
	\end{figure}   

\newpage 

\section{ $Z^\prime$ phenomenonology and current mass constraints }


 $Z^\prime$ boson is predicted in many extensions of the Standard Model (SM). Each with different coupling to the SM fermions. In general generation-dependant couplings of $Z^\prime$ to the SM fermions is given by 
 
 \begin{equation}
	 Z_\mu^\prime \left[ \sum_{i=1,j=1}^{i=3,j=3} \left(g_f^L \right)_{ij} \bar{f}_L^i \gamma^\mu f_L^j + \left(g_f^R \right)_{ij} \bar{f}_R^i \gamma^\mu f_R^j	  \right]
 \end{equation} 
 
 where the indices i and j run over all three generations and f = (u, d, e, $\nu$). The coefficients $\left(g_f^L \right)_{ij}$ and $\left(g_f^R \right)_{ij}$ are real dimentionless parameters, and analogous to the weak hypercharge of elecro-weak interations the gauge charges $z_{fi}^L$, $z_{fi}^R$ determine all the couplings and mass and decay width of $Z^\prime$ \cite{zphysics,zpheno2}. 
    
    
 \begin{itemize}
 	\item Simple U(1) extensions of the standard model, also known as Sequential Standard Model(SSM) \cite{zpheno1}, which assumes $Z^\prime$ couplings to SM fermions to be the same as $Z^0$ but with a polemass of the order of TeV. These theories have the conserved gauge charges (as in column 1) proportional to the baryon number minus $x$ times the lepton number $(B - x L)$, and are in general known as $U(1)_{B-xL}$ extensions of the standard model \cite{zpdg}.
 	
 	\begin{figure}[h]
 		\begin{centering}	
 			\includegraphics[scale=0.5]{./zpcharges.png} 
 			\caption{ Examples of generation-independant U(1) gauge charges for quarks and leptons, here $x$ is any rational number.}
 			\label{U1charges}
 			\centering
 		\end{centering} 		
 	\end{figure}   
 	   
 	$Z^\prime$ is leptophobic for $x=0$, i.e. it only decays to quarks and the interactions preserve baryon number, for $x>>1$ it is quark-phobic. The latter possibilitiy is being ruled out by various collider physics experiments. The case where $x=1$ is particularly important, it gives $U(1)_{B-L}$,  the only anomaly-free global symmetry of the Standard Model, easily promoted to a local symmetry by introducing three right-handed neutrinos, which automatically make neutrinos massive. $B-L$ symmetry remains the only unbroken symmetry of the SM so far \cite{bl}.   	  
 	   
	\item The second column shows charges for $U(1)_{10+x5}$, these symmetry groups are part of $E_6$ Grand Unified Theories(GUT) \cite{zpheno4}. This set leads to $	Z - Z^\prime$ mass mixing close to electroweak scale. Under the third set,$U(1)_{d−xu}$, the weak-doublet quarks are neutral, and the ratio of $u_R$ and $d_R$ charges is $−x$.  For $x= 1$ this is the “right-handed” group $U(1)_R$. 	
 \end{itemize} 
   
   The existance of this $U(1)^\prime$ symmetry will have profound implications of for particle physics and cosmology. The most important implications include an extended Higgs sector, extended neutralino sector, and solution to the $\mu$ problem in supersymmetry, exotic fermions needed for anomaly cancellation, possible flavor changing neutral current effects, neutrino mass, possible $Z^\prime$ mediation of supersymmetry breaking, and cosmological implications for cold dark matter and electroweak baryogenesis \cite{paul}.     
   
\newpage  

	In general $Z^\prime$ couples to all the SM and some of the BSM particles, however the current analysis at LHC are motivated by pure SM contribution in which $Z^\prime$, couples to all the SM fermions, namely the minimal U(1) extensions of the SM. Along these lines, the ATLAS and CMS collaborations have searched for $Z^\prime$ bosons by investigating dilepton and dijet final state in detail. By using dilepton data at 13 TeV, the ATLAS collaboration \cite{atlas} set the mass exclusion limits $M_{Z^\prime}>4.5$ TeV(SSM) and $M_{Z^\prime}′>3.8-4.1 $ TeV (GUT-inspired models), whereas CMS obtained $M_{Z^\prime}>4.0 $ TeV(SSM) and $M_{Z^\prime}>3.5$ TeV (GUT-inspired models) \cite{cms}.  For dijets,  the limits are much milder and read $M_{Z^\prime} > 2.1-2.9$ TeV (ATLAS) \cite{atlasjet} and $M_{Z^\prime} >2.7$ TeV (CMS) \cite{cmsjet}.
	
	The U(1) supersymmetric models however by including supersymmetric decay modes lowers the branching ratios into SM final states and impact the $Z^\prime$ mass limits by $200-300$ GeV \cite{zsusy}. 

\newpage

\section{Top quark reconstruction}

The phenomenology of top quark is driven by it's large mass, which also gives it very short lifetime $\sim 10^{-24}$s and makes it the only quark that decays semi-weakly into W boson and b quark. The brancing ratio for the top decay $t \to W b $ is close to unity, and for this analysis we assume it to be unity. The signature of top quarks which decay hadrnoically i.e. for which the decay $t \to W b $ is followed by $W \to q \bar{q}$ where $q = (u, d, s, c)$, is in general three quark-like jets in the final state from the hadronization of the quarks. If the parent top is at rest then these three jets are well separated from each other in $(y,\phi)$ plane and can in general be identified as thee distinct quark jets in an event. On the other hand is the parent top is boosted i.e. has high $p_T$ then the three quark jets will tend to be collinear along the direction of the parent top, and consenquently instead of having three well separated jets in final state we may get only one  fatjet. Substructure of this fatjet gives information of the corresponding decay. \ref{boost}.     

 	\begin{figure}[h]
 		\begin{centering}	
 			\includegraphics[scale=0.4]{./topboost.png} 
 			\caption{}
 			\label{boost}
 			\centering
 		\end{centering} 		
 	\end{figure}   

To quantify things lets consider the angular separations between partons in the process $ t \to w b \to q \bar{q} b$ as a function of the $p_T^{top}$ fig:\ref{rwbt}. The angular separation in rapidity-azimuth $(y, \phi)$ plane $\left(  \Delta R = \sqrt{\Delta y^2 + \Delta \phi^2} \right)$ between the decay products  in a two-body depends on the mass and boost of the decaying particle and in the high $p_T$ approximation it is given by .

\begin{equation}
 \Delta R \approx \frac{2m}{p_T}
\end{equation} 

Where m and $p_T$ are the mass and transverse momenta of the decaying particle. For a top $(m = 173)$ GeV \cite{pdg-top} with $p_T = 200$ GeV the separation is $\Delta R = 1.73 $, similarly for a W boson $(m = 80)$ \cite{pdg-w} with same $p_T$ the separation is $\Delta R = 0.8$ . Fig \ref{rwbt} shows the relation between $\Delta R $ and $p_T$ for the two body decays of top and w respectively. A falling trend in the angular separation as we go to higher $p_T^{top}$  is apparent, from the plot one can also see that roughly $\Delta R_{wb}^{avg} < 1.7$ for $p_T^{top} > 200$, and $\Delta R_{qq}^{avg}< 0.8 $ for $p_T^{w} > 200$ GeV.   

The top tagging algorithms that are aimed at reconstructing low $p_T$ top work on the assumption that it would show up in an event as one b-tag and two distinct w-tag jets, this assumption grows weaker as $p_T$ of the top increases and the top instead of corresponding to three quark jets which are well separated from each other, corresponds to one fatjet in which all three jets are merged. The substructure of this fatjet contains the information about the undelying decay.  The top reconstruction effeciency of such algorithms is also expected to be going downhill for boosteed tops. We will be looking at the tops in the decay $Z^\prime \to t \bar{t}$. These tops in general may be quite boosted in laboratory frame depending upon the mass of the $Z^\prime$ fig:\ref{toppt} (bottom right). Jet substructure techniques are required to reconstruct top in such cases.  
   


\newpage
 	\begin{figure}[h]
 		\begin{centering}	
 			\includegraphics[scale=0.15]{../pp2zp2tt2qqblv_1000root/partons/rtwb.png} 
 			\includegraphics[scale=0.15]{../pp2zp2tt2qqblv_1000root/partons/rwqqb.png}
 			\caption{A plot between angular separation between decaying particles of the top and w decay, in $(y,\phi)$ plane,  vs $p_T$ of top and w respectively.}
 			\label{rwbt}
 			\centering
 		\end{centering} 		
 	\end{figure}   
 
 
\subsection{Top tagging algorithms}

	A Jet \footnote{By Jet we mean QCD Jet.}is an artifact of all the high energy collider physics experiments, it's a collimated spray of particles in a given direction which can all be traced back to a common vetrex. These particles result from hadronization of a quark therefore they also share the energy and momenta of the original quark. In the experiments these particles show up as a set of hadrons and photons which are concentrated within some annular region of the detector. There is no universally accepted definition of a jet however the most natural definition is based on a cone radius in pseudorapidity-azimuth plane within which most of the jet particles are contained.
	
	A jet algorithm operates on the four momenta of the final state particles or some detector quantities like the energy deposits in calorimeters, these quantities are referred to as 'constituents'of the jet. A jet algorithm clusters two consituents i and j into one new constituent if the quantity $\Delta d_{ij}$ which is related to the angular separation $\Delta R_{ij}$ between the  between the constituents and their transverse momenta , satisfies the condition $d_{ij} < d_{iB}$, i.e. if they are closer to each other than they are to the beam in some distance measure. Clustering is done recursively untill there are no clusters left within R distance from one another. For our analysis we use sequantial anti-$k_T$ and Cambridge/Aachen(C/A) algorithms which differ only in the choice of the distance parameter $d_{ij}$ \footnote{ $d_{ij} = \frac{\Delta R_{ij}}{R}$ and $d_{iB} = 1 $ for C/A and $d_{ij} = min \left( \frac{1}{p_{\text{T,i}}^2}, \frac{1}{p_{\text{T,j}}^2} \right)\times \frac{ \Delta R_{ij}}{R}$ and $ d_{iB} = \frac{1}{p_{T,i}^2} $ for anti-$k_T$. The two consituents i,j are clustered together if $d_{ij} < d_{iB}$.}. More details on jet finding algorithms can be found here \cite{jet-topography}.
	
	\subsubsection{Tagging of low $p_T$ top}
	The top when produced with low momentum can be reconstructed via three angularly resolved jets. These jets have to satisfy certain properties in order for them to be considered as 'top candidates'. Top tagging algorithms for moderately boosted top are based on this property. We describe two such algorithms which expolit this property of low $p_T$ tops and identify the 3-pronged structure by putting mass constraints on the reconstructed jets. Jets which do not satisfy $p_T^{jet} > 20$ GeV are discarded.   
	\begin{enumerate}
		\item Kinematic mass cut
		\item $\chi^2$ minimization. 
	\end{enumerate}		
	 	
	 First algorithms takes reconstructed jets and first identifies a pair of w-tagged jets based on the criterion
	 $| m_{ij} - m_w | < \Delta_w$, where $m_{ij}$ is the invariant mass of jet i and jet j, $\Delta_w$ is some positive number. If two different pairs of jets satisfy this criteria then pair with invariant mass closest to w mass is chosen, and if two different pairs have same invariant mass till two significant digits then the one with highest $p_T$ is chosen to be w candidate. After identifying two jets(i and j) as w tagged the next step is the identification of the triplet (ijk) as top tagged if $|m_{ijk} -  m_{top}| < \Delta_{top}$ . Again the triplet which has $m_{ijk}$ closest to top mass is chosen in case two pairs satisfying this criteria. The process is repeated untill all the jets have been analysed. For this study we take $\Delta_w = 5.0\Gamma_w$ and $\Delta_{top} = 5.0\Gamma_{top}$, where $\Gamma_w = 2.0$ \cite{pdg-w} and $\Gamma_{top} = 1.5$\cite{pdg-top} are the decay widths of w and top respectively. These tagged top jets are then matched with the parton level top quark. Reconstruction efficency is defined to be the fraction of number of parton level top quarks for which a match with tagged top jets  is found within a cone radius of 0.3. 
	 
	 \begin{equation}
	 \epsilon_{\text{top reco}} = \frac{N(R_{ \text{top-quark,  tagged-topjet}} \le 0.3 )}{N( \text{parton level top})}
	 \label{recoeff}	
	 \end{equation}
	 
The $\chi^2$ minimization procceds by identifying the triplet (ijk) from the list of reconstructed jets which gives the minimum value of the quantity  

	\begin{equation}
		\chi_{ijk}^2 = \frac{ \left(m_{ij} - m_{w}^{PDG} \right)^2 }{ \sigma_w^2}	+  \frac{ \left(m_{ijk} - m_{top}^{PDG} \right)^2 }{ \sigma_{top}^2}
		\label{chisq}
	\end{equation}

where $\sigma_w$ and $\sigma_{top}$ are related to the decay widths of w and top respectively. Note that the same combination of jets (ijk)  will give the minimum value for $\chi_{ijk}^2$ if $\sigma_w$ and $\sigma_{top}$ have fixed ratio. For the sake of simplicity we take them to be $\sigma_w = 5.0\Gamma_w$, $\sigma_{top} = 5.0\Gamma_{top}$, and the top tagging criteria becomes $\chi_{ijk}^2 \le 2$. These tagged top jets are again matched with a parton level top and the reocnstruciton effciency is given by equation \ref{recoeff}. 

\subsubsection{Tagging of boosted tops}

Top tagging algorithms for boosted tops make use of the fact that the decay products of top have large fraction on their momenta along the direction of the original top quark and analyse the top decay from geometrically large objects called fatjets. The siz e of this fatjet should be large enough to contain all three decay products of top. Fig \ref{topjetsize} shows the size of the fatjet which can contain the top mass for a given top $p_T$. It's clear that for tagging high $p_T$ tops we should chooose smaller fatjet size. Many algorithms are available for tagging of boosted tops such as mass drop tagger, Jon Hopkins tagger etc. , details about these algorithms can be found here \cite{toptaggin}. For our analysis we use HepTopTagger(Heidelberg-Eugene-Paris) with C/A fatjets of radius $R_{\text{fat}}=1$.   

\begin{figure}[h]
	\begin{centering}	
		\includegraphics[scale=0.15]{../pp2zp2tt2qqblv_1000root/partons/topjetsize.png} 
		\caption{Size of fatjet around top which can contain the top mass vs $p_T^{top}$.}
		\label{topjetsize}
		\centering
	\end{centering} 		
\end{figure}    

HepTopTagger algorithms procceds in the following way \cite{heptop}.

\begin{enumerate}
	\item Reconstruct C/A fatjets with R=1 and $p_T^{jet} > 200$ GeV. 
	\item Undo the last stage of clustering into jets $j_1$ and $j_2$ with $m_{j_1} > m_{j_2}$. Keep the two jets if they satisfy the mass drop criteria $m_{j_1} < 0.8 m_j$ . Otherwise keep only $j_1$. This declustering is done iteratively untill we have subjets with $m_{j_i} > 30$ GeV.
	\item Construct exactly three subjets (123) from five hardest subjets. Accept them as top candidates if their invariant masses $(m_{12}, m_{13}, m_{23})$ satisfy one the following criteria. 
	\begin{eqnarray*}	
	0.2 < \tan^{-1} \left(\frac{m_{13}}{m_{12}} \right) \text{  and } R_{min} < \frac{m_{23}}{m_{123}} < R_{max} \\
	R_{min}^2 \left( 1 + \left( \frac{m_{13}}{m_{12}} \right)^2 \right) < 1 - \left(\frac{m_{23}}{m_{123}} \right)^2 < R_{max}^2 \left( 1 + \left( \frac{m_{13}}{m_{12}} \right)^2 \right) \text{  and } \frac{m_{23}}{m_{123}} > 0.35\\
	R_{min}^2 \left( 1 + \left( \frac{m_{12}}{m_{13}} \right)^2 \right) < 1 - \left(\frac{m_{23}}{m_{123}} \right)^2 < R_{max}^2 \left( 1 + \left( \frac{m_{12}}{m_{13}} \right)^2 \right) \text{  and } \frac{m_{23}}{m_{123}} > 0.35
	\end{eqnarray*}
	Where $R_{min} = 0.85 \frac{m_w}{m_{top}}$ and $R_{max} = 1.15 \frac{m_w}{m_{top}}$ 
	\item Require the resultant $p_T$ of the three subjets to be greater than 200 GeV.
\end{enumerate}

In order to have a comparision between the three algorithms, we require the fatjets to be top tagged in the mass range $m_{top} \pm \Delta_{top}$ and w-tagged in the mass range $m_{w} \pm \Delta_{w}$. Then as a consistancy check these tagged jets are matched with the corresponding parton level top, and the efficiency is defined same as equation \ref{recoeff}.
     

\subsection{Single top reconstruction}

In order to compare the top reconstruction efficiencies of the algorithms we generated $5 \times 10^5$ events of the type  $Z^\prime \to t t \to \text{wb wb } \to q_1 \bar{q}_2\text{b }l \nu_l\text{b}$, to Leading Order(LO) using $\text{MADGRAPH}_5$aMC@NLO \cite{mad1,mad2} for $M_{Z^\prime} = 500,1000,2000$ GeV. PYTHIA8.2 \cite{pythia} is used for further decays and hadronization  of these particles and for evolution of QCD showers. Final state of the signal contains 4 jets, one lepton and missing energy. We select events in which there are atleast three AK5(anti-kT with R=0.5) jets and one lepton each with $p_T > 20$ GeV and $|\eta| < 2.5$. The jets are groomed using SoftDrop\cite{softdrop} which removes soft componenent of the radiation. Top jets are tagged using kinematic mass cut, and HepTopTagger in the Top and W mass range $m_{top} \pm \Delta_{top}$, $m_{w} \pm \Delta_{w}$ respectively, and with $\chi^2$ minimization where $\chi_{min}^2 < 2$ is the tagging criteria. Fig \ref{topm} shows the invariant mass distributions of the tagged top jets. We can see that the tagging efficiency of HepTopTagger for higher $M_{Z^\prime}$ is higher while for the other two methods the opposite is true, due to the larger boost of the top quark.\\
\linebreak

The quantity $\Delta R( top^{tagged}, top^{parton} ) =  \sqrt{ \left(  y_{top^{tagged}} - y_{top^{parton}} \right) ^2 +  \left( \phi_{top^{tagged}} - \phi_{top^{parton}} \right) ^2 }$ which is the angular separation between the tagged top jets and the corresponding parton level top gives a measure of precision of the reconstructed momentum of top tagged jets \footnote{Here by momemtum we mean 3 dimentional vector momentum $\vec{P}_{top}$.  }. For a top tagged jet to have reconstructed the parton level top we require it to have $\Delta R < 0.3$ with the corresponding parton level top. Fig \ref{delR} shows the distribution of $\Delta R$. The tagged top jets which also match with the corresponding parton level top are called reconstructed top jet. Fig \ref{toppt} shows the $p_T$ distribution for reconstructed top jets. For moderately boosted top quarks $(pT \approx 500 GeV)$ , conventional top quark reconstruction methods, which exploit the decay chain topology, remain adequately efficient whereas the reconstruction of high $p_T$ tops is better for HepTopTagger .The top reconstruction efficiency as a function of $p_T^{top}$ is shown in fig.\ref{recoeff_vs_pt}.       
 

\newpage 

\begin{figure}[h]
	\begin{centering}	
		\includegraphics[width=20cm,height=7cm]{../pp2zp2tt2qqblv_2000root/jets/TagParticles12_M.png} 
		\caption{ Invariant mass distribution of the tagged top jets in the process  $Z^\prime \to t \bar{t} \to qqbl\nu_{l}b$ using kinematic mass cut(top left), $\chi^2$ minimization (top right), HepTopTagger (bottom left) and of the parton level top quark (bottom right). For $M_Z^\prime$ = 0.5 TeV(black), 1.0 TeV(red), 2.0 TeV(green)  respectively. }
		\label{topm}
		\centering
	\end{centering} 		
\end{figure}

\begin{figure}[h]
	\begin{centering}	
		\includegraphics[width=20cm,height=7cm]{../pp2zp2tt2qqblv_2000root/jets/RecoParticles12_pT.png} 
		\caption{ $p_T$ distribution of the reconstructed tops viz tagged top jets which have also matched with it's parton level top, in the process  $Z^\prime \to t \bar{t} \to qqbl\nu_{l}b$ using kinematic mass cut(top left), $\chi^2$ minimization (top right), HepTopTagger (bottom left) and of the parton level top quark (bottom right). For $M_Z^\prime$ = 0.5 TeV(black), 1.0 TeV(red), 2.0 TeV(green)  respectively. Reconstruction efficiency is defined as fraction of input top quarks that are reconstructed. }
		\label{toppt}
		\centering
	\end{centering} 		
\end{figure}   
\newpage

\begin{figure}[h]
	\begin{centering}	
		\includegraphics[width=8cm,height=3.5cm]{../pp2zp2tt2qqblv_500root/jets/delR.png} 
		\includegraphics[width=8cm,height=3.5cm]{../pp2zp2tt2qqblv_1000root/jets/delR.png} 
		\includegraphics[width=8cm,height=3.5cm]{../pp2zp2tt2qqblv_2000root/jets/delR.png}
		\caption{ $\Delta R$ between the tagged top jets and the parton level top quark. For $M_Z^\prime = 500$GeV (top left), $M_Z^\prime = 1000$ GeV (top right), $M_Z^\prime = 2000$ GeV (bottom). Vertical black line in all the histograms denote the value $\Delta R = 0.3$. A top is termed 'reconstructed' if the tagged top jet has $\Delta R < 0.3$  with the corresponding parton level top i.e. left of the line.}
		\label{delR}
		\centering
	\end{centering} 		
\end{figure}   

\begin{figure}[h]
	\begin{centering}	
		\includegraphics[width=8.5cm,height=4cm]{../pp2zp2tt2qqblv_500root/jets/Recoeff4_top_pT.png} 
		\includegraphics[width=8.5cm,height=4cm]{../pp2zp2tt2qqblv_1000root/jets/Recoeff4_top_pT.png} 
		\includegraphics[width=8.5cm,height=4cm]{../pp2zp2tt2qqblv_2000root/jets/Recoeff4_top_pT.png}
		\caption{Top reconstruction efficency vs $p_T^{top}$. For $M_Z^\prime = 500$GeV (top left), $M_Z^\prime = 1000$ GeV (top right), $M_Z^\prime = 2000$ GeV (bottom).  }
		\label{recoeff_vs_pt}
		\centering
	\end{centering} 		
\end{figure}   

\newpage
\subsection{QCD background}

 	

 Without a mass cut, the QCD jetbackground swamps the hadronic top signal by orders ofmagnitude. The most basic tagging method after giving upconventional methods is to use the jet mass as a discrim-inator between the QCD background and the hadronic topsignal; the high-pTtop-jet mass distribution should peakaround the top mass, while the QCD jet mass distributionpeaks near zero

 However, using a jet mass as a discrim-inator is more complicated for several reasons. Because ofradiation, QCD jets acquire a large tail in the mass distri-bution. The cross section for acquiring large jet mass, forexample, near the top mass, increases substantially withpTand cone size. Top jets also broaden due to radiation,hardening their jet mass distribution.3Furthermore, a finitejet reconstruction cone size will not always capture all thedaughters of the top quark decay chain, thus softening itsmass distribution. The net effect is a smearing of theexpected naive, broadened, Breit-Wigner distribution forthe top-jet mass distribution.
 
 With the theoretical machinery discussed in the previoussection, we are able to make a prediction of the rate atwhich QCD jets will fake the mass signature of top jets. Wedefine the fractional fake rate as the fraction of jets with140 GeV mJ 210 GeV, for givenpTandR. 

\newpage

\section{Event Selection}
\newpage
\section{4 Top QCD background}
\newpage
\section{Results}
\newpage
\section{Conclusion}
\newpage
\section{Aknowledgement}	 
	\newpage 

   Top jets are Jet reconstruction techniques are used in order to reconstruct the properties of the top     

We also look at dominant background for this process, which is QCD.  Current LHC energy $\sqrt{s} = 14$ is assumed. 


Out of which one top anti-top quark pair( which came from $Z^\prime$ decay ) is expected to be boosted more than the other  


 
expected to be greater than 209 GeV .


  All currentZ′analyses at the LHC are however guided by non-supersymmetric considerations in whichtheZ′boson only decays into SM particles  

Z′bosons with couplings to quarks maybe produced at hadron colliders in the s-channel and would show up as resonances in the invariant mass distribution of the decay product
 

with  the  same  couplings  as  the  SMZ0but a much higher polemass (on the order of TeV).     

\newpage
\begin{thebibliography}{}
	\bibitem{paul} Langacker P., \href{https://arxiv.org/abs/0801.1345v3}{The Physics of Heavy $Z^\prime$ Gauge Bosons}, arXiv:0801.1345, 7-16
	
	\bibitem{zpdg}  B.A. Dobrescu (Fermilab) and S. Willocq(Univ. of Massachusetts), \href{http://pdg.lbl.gov/2017/reviews/rpp2017-rev-zprime-searches.pdf}{$Z^\prime$-Boson Searches}, September 2017.
	
	\bibitem{atlas} The ATLAS Collaboration, \href{https://arxiv.org/pdf/1707.02424.pdf}{Search for new high-mass phenomena in the dilepton final state using 36 $fb^{−1}$ of proton–protoncollision data at $\sqrt{s}=13$ TeV with the ATLAS detector.}, arXiv:1707.02424v2  [hep-ex]  15 Nov 2017.
	
	\bibitem{cms} CMS Physics Analysis Summary, \href{http://inspirehep.net/record/1479666/files/EXO-16-031-pas.pdf?version=1}{Search for a high-mass resonance decaying into a dilepton final state in 13 $fb^{−1}$ of pp collisions at $\sqrt{s}=13$ TeV.}, 2016/08/05
	  
	\bibitem{atlasjet} ATLAS Collaboration, \href{https://journals.aps.org/prd/pdf/10.1103/PhysRevD.96.052004 }{Search for new phenomena in dijet events using $37fb^{-1}$ of pp collision data collected at $\sqrt{s}=13$ TeV with the ATLAS detector.}, 28 March 2017.  
	
	\bibitem{cmsjet} The CMS Collaboration, \href{https://www.sciencedirect.com/science/article/pii/S0370269317301028?via%3Dihub}{Search for dijet resonances in proton–proton collisions at and constraints on dark matter and other models},14 February 2017 
		
	\bibitem{zpheno1} R. Foot(a),X.-G. He1(b),H. Lew2(c)and R. R. Volkas3(d), \href{https://arxiv.org/pdf/hep-ph/9401250.pdf}{Model for a LightZ′Boson}, arXiv:hep-ph/9401250v1, 13 Jan 1994	
	
	\bibitem{zphysics} F. DEL AGUILA,  \href{https://arxiv.org/pdf/hep-ph/9404323.pdf}{The Physics of $Z^\prime$ bosons},Departamento de F ́ısica Te ́orica y del Cosmos, Universidadde GranadaGranada, 18071, Spain, arXiv:hep-ph/9404323v1  22 Apr 1994
	
	\bibitem{zpheno2} Thomas G. Rizzo, \href{https://arxiv.org/pdf/hep-ph/0610104.pdf}{$Z^\prime$ Phenomenology and the LHC},  Stanford Linear Accelerator Center,2575 Sand Hill Rd., Menlo Park, CA, 94025, arXiv:hep-ph/0610104v1,  9 Oct 2006
	
	\bibitem{zpheno3} Joseph D. Lykken, \href{https://arxiv.org/pdf/hep-ph/9610218.pdf}{$Z^\prime$ Bosons and Supersymmetry} , Theoretical Physics Dept., MS106Fermi National Accelerator Laboratory, arXiv:hep-ph/9610218v3  25 Oct 1996
	
	\bibitem{zpheno4} David London, Jonathan L.Rosne \href{https://journals.aps.org/prd/pdf/10.1103/PhysRevD.34.1530}{Extra gauge bosons in $E_6$}, 01/09/1986
	
	\bibitem{bl} Julian Heeck, \href{https://arxiv.org/pdf/1408.6845.pdf}{Unbroken B − L Symmetry}, arXiv:1408.6845v2 10 Nov 2014, Phys. Lett. B739, 256–262 (2014) 
	
	\bibitem{loophole} Jack Y. Araza, Gennaro Corcellab, Mariana Frankaand Benjamin Fuksc,d,e , \href{https://arxiv.org/pdf/1711.06302.pdf}{Loopholes in $Z^\prime$ searches at the LHC:exploring supersymmetric and leptophobic scenarios}, arXiv:1711.06302v2, 16 Feb 2018 
	
	\bibitem{zsusy} Gennaro Corcella, \href{https://arxiv.org/pdf/1307.1040.pdf}{Searching for supersymmetry in $Z^\prime$ decays}, arXiv:1307.1040v2, 5 Jul 2013
	
	\bibitem{pdfsets} \href{https://lhapdf.hepforge.org/pdfsets}{LHAPDF 6.2.3}
	
	\bibitem{blref1} \href{https://arxiv.org/abs/1811.11452}{arXiv:1811.11452}
	
	\bibitem{blref2} \href{https://arxiv.org/abs/1804.04075}{arXiv:1804.04075}
	
	\bibitem{blref3} \href{https://arxiv.org/abs/0812.4313}{Phenomenology of the minimal B-L extension of the Standard model: Z' and neutrinos}, arXiv:0812.4313
	
	\bibitem{mad1} \href{https://arxiv.org/abs/1405.0301}{arXiv:1405.0301}
	
	\bibitem{mad2}
	\href{https://arxiv.org/abs/hep-ph/0204244}{arXiv:hep-ph/0204244}
	
	\bibitem{pdg-top} \href{http://pdg.lbl.gov/2019/tables/rpp2019-sum-quarks.pdf}{Quarks data in PDG 2019}
	
	\bibitem{pdg-w} \href{http://pdg.lbl.gov/2019/listings/rpp2019-list-w-boson.pdf}{W boson data in PDG 2019}
	
	\bibitem{jet-topography} G. P. Salam, Towards Jetography \href{https://arxiv.org/abs/0906.1833}{arXiv:0906.1833v2}
	
	\bibitem{anti-kt} Matteo Cacciari, Gavin P. Salam, Gregory Soyez, \href{https://arxiv.org/abs/0802.1189}{The anti-k_t jet clustering algorithm} 
	
	\bibitem{toptaggin} Tilman  Plehn1and, Michael Spannowsky \href{https://arxiv.org/pdf/1112.4441.pdf}{Top Tagging}
	
	\bibitem{heptop} Tilman Plehn,1Michael Spannowsky,2Michihisa Takeuchi,1and Dirk Zerwas \href{https://arxiv.org/pdf/1006.2833.pdf}{Stop Reconstruction with Tagged Tops}, arXiv:1006.2833v2 [hep-ph] 31 Aug 2010
	
	\bibitem{pythia} Present program version: \href{http://home.thep.lu.se/~torbjorn/Pythia.html}{PYTHIA8.2}
	
	\bibitem{softdrop} \href{https://phab.hepforge.org/source/fastjetsvn/browse/contrib/contribs/RecursiveTools/tags/2.0.0-beta2/}{SoftDrop 2.0.0 }
	
	\bibitem{topjetLHC} \href{https://journals.aps.org/prd/pdf/10.1103/PhysRevD.79.074012}{Top quark jets at the LHC}
\end{thebibliography} 


\end{document}












